\section*{Preliminaries and notation}
\begin{itemize}
    \item $S \subseteq \zo^*$ defines set $S$ as a finite subset of $\zo$ all finite-length strings. 
    \item $x \in_R S$, $R$ indicates x is chosen randomly / uniformly from $S$
    \item $U_n$ x is chosen from the set of all n-bit strings
    \item $\mu(\cdot)$ means the negligible function can take any input. Negligible functions decrease faster than inverse polynomial as $n$ increases
    \item must use positive polynomial, if not, maybe it won't be negligible. We want $\mu(n) < \frac{1}{p(n)}$
    \item $\lambda$ is an empty string
\end{itemize}


$$
X = \{X(a,n)\}_{a\in\zo^*; n \in \mathbb{N}}
$$

"A probability ensemble" is a collection or family of random variables, denoted by $X = \{X(a,n)\}$ i.e. $X$ represents the set of $\{X(a,n)\}$. 
\begin{itemize}
    \item Set $X = \{X(a,n)\}$ defines the set of random variables $X$
    \item Set Subscript ${a\in\zo^*; n \in \mathbb{N}}$ defines the indexing, that is, exactly what values of $a, n$ can take for infinitely any element in set $X$. e.g. $X('01', 3)$ is the index of a random variable $X$ where $a$ is a binary string of any finite length "0, 1, 01, 000", $n$ is a natural number 1,2
    \item $\zo^*$ is the Kleene star operation, means all finite strings, an infinite set because there's no limit to the length
    \item $n\in\Nat$ means $n$ can be any natural number, also an infinite set
    \item indexing gives us an address or a way to talk about 1 specific random variable rather than the collection
    \item "probability ensemble" is a term in cryptography / probability theory referring to a collection or family of probability distributions or random variables. Used to describe systems where behaviours depend on on input length, security parameter, etc
    \item "ensemble" means we're dealing with a collection of probabilistic objects rather than a single fixed distribution
\end{itemize}

$$
\abs{
    \Pr\[D(X(a,n)) = 1 \] - \Pr\[D(Y(a,n))=1\]
} \leq \mu(n)
$$

The (absolute difference) of the probability that the distinguisher $D$ outputs 1 when given a sample from $X(a,n)$ and the probability $D$ outputs 1 when given a sample from $Y(a,n)$ is $\leq$ the negl. function $\mu(\cdot)$ 

Notes:
\begin{itemize}
    \item $D$ is a $\ppt$ algorithm trying to distinguish between samples from $X$ and $Y$, not guessing the bit
    \item $D's$ output is binary $\zo$. Output 1 means "The sample came from $X$" or "This is a real sample" i.e. distinguish
    \item We look at the difference in probability of $D$ outputting 1 for $X$ vs $Y$
    \item a negl. difference means $D$ can't distinguish
    \item absolute value || means we only want the magnitude of the difference, doesn't matter which is larger
    \item $\mu(n)$ being a negl. function means the difference becomes smaller as the security parameter grows larger
\end{itemize}

