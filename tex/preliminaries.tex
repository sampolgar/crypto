
\section{PRF/VRF}
\subsubsection{Dodis-Yampolskiy PRF Standard Version} 
$s \in \Z_p$ is the secret key, $pk = [s]_1$ is the $\G_1 $ point. $x$ is the PRF input.
\[\prover_s(x) \rightarrow y = e(g,[x+s]^{-1}), \vk = [x + s]^{-1}\]
\[\verifier(x,y,\vk): e(x\cdot pk, \vk) = e(g,g)  \quad \text{ and } \quad y = e(g,\vk)\]
\[ = e([x + s], [x + s]^{-1}) = e(g,g) \quad \text{ and } \quad e(g,[x+s]^{-1}) = e(g,[x + s]^{-1})\]
Explanation: the verifier has $x, y, \vk, pk$ and computes $e(pk \cdot x, \vk) = e(g,g)$, if correct, the $\vk$ given correctly verifies the $pk$ and $x$. The second pairing $y = e(g,\vk)$ verifies the prover has not modified $y$ or $\vk$ to pass the first pairing check.

\subsubsection{Dodis-Yampolskiy PRF Private Version} 
$\nullif = \PRF_s(\sn) = [\secret + \sn]_h^{-1}$ where h $\getsr$ $\G_1$ is a fixed $\G_1$ public parameter and $(\tilde{h}, \tilde{w} \getsr \G_2^2)$ are fixed $\G_2$ public parameters for the $\PRF$, $t$ is a secret PRF randomizing factor $\in \Z_p$.
\[
    \begin{aligned}
    e(\nullif, \vk) &\stackrel{?}{=} e(\nullif, [s + sn]_{\tilde{h}} \cdot [t]_{\tilde{w}}) \\[5pt]
    &= e(\nullif, [s + sn]_{\tilde{h}}) \cdot e(\nullif, [t]_{\tilde{w}}) \\[5pt]
    &= e([s + sn]_{\tilde{h}}^{-1}, [s + sn]_{\tilde{h}}) \cdot y \stackrel{?}{=} e(h, \tilde{h}) \cdot y\\[2pt]
\end{aligned}
\]
Soundness:
\[\nullif = [s + sn]_h^{-1}\]
\[\vk = [s + sn]_{\tilde{h}}\cdot [t]_{\tilde{w}} \]
\[ y = e(\nullif, \tilde{w})^t \]
Verification, assuming correctness of $\vk$ and $y$:
\[
    \begin{aligned}
    e(\nullif, \vk) &\stackrel{?}{=} e(h,\tilde{h}) \cdot y \Leftrightarrow \\
    e(\nullif, [s + \sn]_{\tilde{h}}\cdot [t]_{\tilde{w}} ) &= e(h,\tilde{h}) \cdot e(\nullif, \tilde{w})^t \Leftrightarrow \\
    e(\nullif, [s + \sn]_{\tilde{h}}) \cdot e(\nullif,[t]_{\tilde{w}}) &= e(h,\tilde{h}) \cdot e(\nullif, \tilde{w})^t \\
\end{aligned}
\]
Soundness: we can show, when $s \neq \sn$, $\nullif = [s+\sn]_h$. Assume $\nullif = [a]_h$ for some $a \in \Z_p$, then 
\[
    \begin{aligned}
    e([a]_h, [s + \sn]_{\tilde{h}}) \cdot e([a]_h,[t]_{\tilde{w}}) &= e(h,\tilde{h}) \cdot e([a]_h, \tilde{w})^t \Leftrightarrow\\
    e([a]_h, [s + \sn]_{\tilde{h}}) \cdot e([a]_h,[t]_{\tilde{w}}) &= e(h,\tilde{h}) \cdot e([a]_h,[t]_{\tilde{w}}) \Leftrightarrow\\
    e([a]_h, [s + \sn]_{\tilde{h}})) &= e(h,\tilde{h}) \Leftrightarrow\\
    a(s + \sn) = 1\\
    a = 1/(s + \sn)
\end{aligned}
\]

Zero-knowledge. "there exists a simulator that given any $\nullif \in \G_1$ can simulate a proof as
\begin{itemize}
    \item Pick random $\vk \in \G_2$
    \item Let $y = e(\nullif, \vk) / e(h,\tilde{h}) \in \Gt$
    \item Simulate a $\sum$-protocol proof that argues correctness of $\vk$ and $y$
\end{itemize}

\newpage

$\ccm$ = coin commitment \\
$\pid$ = user key \\
$\sn$ = coin serial number \\
$r$ = ccm commitment randomness \\
$\rcm$ = registration credential commitment, commitment to $\pid$ and $\PRF$ key $\secret$ \\
$\secret$ = PRF key \\
$a$ = commitment randomness \\
$\nullif$ \\
$\vcm$ = value commitment \\
$z$ randomness in value commitment\\
$v$ = value of coin in commitment

\noindent The goal is to prove the correctness of $\nullif$ and $\vcm$ by proving in zero-knowledge the relation 
\[ \Rr(\ccm, \rcm, \nullif, \vcm, \pid, \sn, v, r, s, a, z ) \text{ holds:}\] 
\begin{equation}
\left(
    \begin{aligned}
        \ccm &= \commitment([ \pid, \sn, v]; r) \quad & \wedge \quad \\[2pt]
        \rcm &= \commitment([ \pid, \secret]; a) \quad & \wedge \quad \\[2pt]
        \nullif &= \PRF_s(\sn) \qquad & \wedge \quad \\[2pt]
        \vcm &= \commitment([0,0,v]; z)
    \end{aligned}
\right)
\end{equation}

In different notation:

\subsubsection{$\Rr_{split}, \pi^{split}$}s
We want to prove this relation and generate the proof because this proves the correctness of $\nullif$ in $\ZK$ as part of the $\Rr_{split}$ relation.
Split refers to the user splitting coins to spend them which I don't need to worry about, but the key thing here is proving $\nullif$ and the relation.
\begin{equation*}
 \text{ZK-PoK}\left\{
    \begin{aligned}
        (\ccm, & \rcm, \nullif, \vcm, \pid, \sn, v, r, s, a, z ) : \\[2pt]
        \ccm &= \commitment([ \pid, \sn, v]; r) \quad & \wedge \quad \\[2pt]
        \rcm &= \commitment([ \pid, \secret]; a) \quad & \wedge \quad \\[2pt]
        \nullif &= \PRF_s(\sn) \qquad & \wedge \quad \\[2pt]
        \vcm &= \commitment([0,0,v]; z) \qquad & \wedge \quad \\[2pt]
    \end{aligned}
\right\}
\end{equation*}

\noindent To prove the above holds in zero knowledge, rather than proving $\Rr_{split}$ directly, we prove a new relation $\Rr^*_{split}$, generate $\pi^*_{split}$, which no longer directly argues $\nullif$ but includes $\vk, y$

\subsubsection{$\Rr^*_{split}, \pi^{*split}$}
\begin{equation*}
 \text{ZK-PoK}\left\{
    \begin{aligned}
        (\ccm, & \rcm, \nullif, \vcm, \pid, \sn, v, r, s, a, z, \vk, y, t ) : \\[2pt]
        \ccm &= \commitment([ \pid, \sn, v]; r)  \textcolor{red}{= g_1^{\pid}g_2^{\sn}g_3^vg^r} \quad & \wedge \quad \\[2pt]
        \rcm &= \commitment([ \pid, \secret]; a) \textcolor{red}{=g_1^{\pid}g_5^sg^a} \quad & \wedge \quad \\[2pt]
        &\textcolor{red}{\cancel{\nullif = \PRF_s(\sn)}} \qquad & \wedge \quad \\[2pt]
        \textcolor{red}{\vk} &= \textcolor{red}{[s + sn]_{\tilde{h}}\cdot [t]_{\tilde{w}}} \qquad & \wedge \quad \\[2pt]
        \textcolor{red}{y} &= \textcolor{red}{e(\nullif, \tilde{w})^t} \qquad & \wedge \quad \\[2pt]
        \vcm &= \commitment([0,0,v]; z) \textcolor{red}{=g_3^vg^z} \qquad & \wedge \quad \\[2pt]
    \end{aligned}
\right\}
\end{equation*}

\noindent Verification: the verifier has $(\nullif, \vk, y)$ and first verifies the correctness of the $\PRF$ with the pairing, then the $\sum$-protocols.

\paragraph{Correctness of $\PRF$ given $(\nullif, \vk, y)$}
Recall $\nullif = [s + sn]_h^{-1}, \vk = [s + sn]_{\tilde{h}}\cdot [t]_{\tilde{w}}, y = e(\nullif, \tilde{w})^t $
\Note{explain the below, why it works}
\[
    \begin{aligned}
    e(\nullif, \vk) &\stackrel{?}{=} e(\nullif, [s + sn]_{\tilde{h}} \cdot [t]_{\tilde{w}}) \\[5pt]
    &= e(\nullif, [s + sn]_{\tilde{h}}) \cdot e(\nullif, [t]_{\tilde{w}}) \\[5pt]
    &= e([s + sn]_{\tilde{h}}^{-1}, [s + sn]_{\tilde{h}}) \cdot y \stackrel{?}{=} e(h, \tilde{h}) \cdot y\\[2pt]
    &\text{we have just proved that} \\
    e(\nullif, \vk) &= e(h, \tilde{h}) \cdot y\\[5pt]
    &\text{that's why we need y, to prove the randomness t} \\
\end{aligned}
\]

How can we prove this part of the identity
Commitment is message + randomness
The message part can be cancelled with each other
All that's left is the randomness
The randomness is still the original randomness (because it's with y)
The nullified contains other pure message


\noindent Sigma protocol \\
\begin{enumerate}
    \item for $\ccm$ prove opening of the commitment $\pid$ $\ZKSoK(\pid)$ 
    \item for $\rcm$ prove opening of the commitment $\pid$ $\ZKSoK(\pid)$
    \item prove equality of $\pid$ in $\ccm$ and $\rcm$
    \item extract $s + \sn$ to a new commitment $\cm = g$
    \item prove $\sn$ in $\ccm$ is in $\vk$ and prove $s$ from $\rcm$ is in $\vk$
    \item prove that the pairing equation worked? 
    \item prove t
\end{enumerate}

\begin{equation*}
    \begin{aligned}
        \ccm &= g_1^{\pid}g_2^{\sn}g^r \quad & \wedge \quad \\[2pt]
        \rcm &= g_1^{\pid}g_3^sg^a \quad & \wedge \quad \\[2pt]
        \cm &= g_1^{\pid}g_3^sg^a \quad & \wedge \quad \\[2pt]
        \ccm(g_1^{\pid}) &= \rcm(g_1^{\pid})
    \end{aligned}
\end{equation*}

\pcb{
\textbf{ Alice} \< \< \textbf{ Bob}  \\[0.1\baselineskip][\hline]
 \<\< \\[-0.5\baselineskip]
 \pid, \sn, s \in \ZZ_q \<\< \\
 r, a \gets \ZZ_q^2\<\< \\
 t_1,\dots, t_5 \gets \ZZ_q^5\<\< \\
 T_1 = g_1^{t_1}g_2^{t_2}g^{t_3} \<\< \\
 T_2 = g_1^{t_1}g_4^{t_4}g^{t_5} \< \sendmessageright*{\ccm, \rcm, T_1, T_2} \< \\
 \< \sendmessageleft*{e} \< e \sample \ZZ\\
    z_1 = t_1 + e \cdot \pid  \<\< \\
    z_2 = t_2 + e \cdot \sn  \<\< \\
    z_3 = t_3 + e \cdot r  \<\< \\
    z_4 = t_4 + e \cdot s  \<\< \\
    z_5 = t_5 + e \cdot a  \<\< \\
 \< \sendmessageright*{z_1, \dots, z_5} \< \\
 \<\< g_1^{z_1}g_2^{z_2}g^{z_3} \stackrel{?}{=} (\ccm)^e \cdot T_1\\
 \<\< g_1^{z_1}g_4^{z_4}g^{z_5} \stackrel{?}{=} (\ccm)^e \cdot T_2\\
}

Completeness: 
\begin{equation}
    \begin{aligned}
        g_1^{z_1}g_2^{z_2}g^{z_3} \stackrel{?}{=}& (\ccm)^e \cdot T_1\\
        g_1^{t_1 + e \cdot \pid}g_2^{t_2 + e \cdot \sn}g^{t_3 + e \cdot r} {=}& (g_1^{\pid}g_2^{\sn}g^r)^e \cdot g_1^{t_1}g_2^{t_2}g^{t_3}\\
        g_1^{t_1 + e \cdot \pid}g_2^{t_2 + e \cdot \sn}g^{t_3 + e \cdot r} {=}& g_1^{\pid \cdot e}g_2^{\sn \cdot e}g^{r\cdot e} \cdot g_1^{t_1}g_2^{t_2}g^{t_3}\\
        g_1^{t_1 + e \cdot \pid}g_2^{t_2 + e \cdot \sn}g^{t_3 + e \cdot r} {=}& g_1^{t_1 + e \cdot \pid}g_2^{t_2 + e \cdot \sn}g^{t_3 + e \cdot r}\\
    \end{aligned}
\end{equation}

\begin{equation}
    \begin{aligned}
        g_1^{z_1}g_4^{z_4}g^{z_5} \stackrel{?}{=}& (\rcm)^e \cdot T_2 \\
        g_1^{t_1 + e \cdot \pid}g_4^{t_4 + e \cdot s}g^{t_5 + e \cdot a} =& (g_1^{\pid}g_3^sg^a)^e \cdot g_1^{t_1}g_4^{t_4}g^{t_5}\\
        g_1^{t_1 + e \cdot \pid}g_4^{t_4 + e \cdot s}g^{t_5 + e \cdot a} =& g_1^{\pid \cdot e}g_3^{s \cdot e}g^{a \cdot e} \cdot g_1^{t_1}g_4^{t_4}g^{t_5}\\
        g_1^{t_1 + e \cdot \pid}g_4^{t_4 + e \cdot s}g^{t_5 + e \cdot a} =& g_1^{t_1 + e \cdot \pid}g_4^{t_4 + e \cdot s}g^{t_5 + e \cdot a}\\
    \end{aligned}
\end{equation}
Because the proof $z_1$ contains $\pid$ and is used in both $\ccm$ and $\rcm$ we can tell with high probability that $\pid$ is consistent.

To ensure txn non-malleability, only Alice who knows the PRF key $s_A$ associated with $\pid_A$ should be able to spend coins. An adversary should not be able to $maul$ Alice's txn to change recipients or amounts. We ensure this by using a \textit{zero-knowledge signature of knowledge (ZKSoK)}, rather than $ZKAoK$ As a result, each input's splitproof $\pi^{*split}_i$ implicitly signs all the outputs which proves Alice is authorized to spend the input coin and prevents txn $mauling$. The signing key can be thought of as Alice's $\PRF$ key $s_A$ and all other secrets in the relation $\Rr_{split}$ which only she knows.

Notation
When proving a relation $\Rr_{rel}(\zkx; \zkw)$ holds, we use $\ZKProve_{rel}(\zkx; \zkw)$ and $\ZKVerify_{rel}(\zkx)$ as notation.

\section{ZKP}
% https://claude.ai/chat/da807505-abbb-4e01-afde-a2d977e652ce
\[\Lang = \left\{((U, V, U', V'),(\alpha)) : V = U^{\alpha} \wedge V' = U'^{\alpha} \right\} \subseteq G^2 \times (G')^2 \times \Z_q\]

\[\Lang = \left\{((\text{Instance of the problem}),(\text{Witness})) : \Rr \right\} \subseteq G^2 \times (G')^2 \times \Z_q\]

\noindent A relation $\Rr$ is a set of pairs $(x,w)$ that satisfy a relationship. E.g. the relationship $\Rr_{squared}$ = $\{((x),w : x=w^2)\}$ consists of all pairs $(x,w)$ where $x$ is the square of $w$. The language $\Lang_{squared}$ is defined as $\Lang_{squared} = \{x  : \exists w \text{ such that }(x,w) \in \Rr_{squared} \}$
$\Lang$ includes all instances $x$ where there exists a witness $w$ such that $(x,w)$ satisfies the relation $\Rr$


\noindent A language $\Lang$ of relations $\Rr$ i.e. $\Lang(\Rr)$ is a set of tuples $(x,w)$ where $x$ is a public input and $w$ is a witness (a private input) such that a certain relation $\Rr$ between $x$ and $w$ holds. In this case, the $(U, V, U', V')$ are the public parameters, $\alpha$ is the witness. $\Lang$ consists of all tuples $((U, V, U', V'),\alpha)$ where the relation $V = U^{\alpha}, V'=U'^{\alpha}$ holds.

\noindent Prover's Relation: The prover has a witness $\alpha \in \Z_q$ and secret randomness $r \getsr \Z_q$. Prover's relation is defined by $\Rr := (U^r, (U')^r) \in G \times G' $ 

\subsubsection{Sigma Proofs}

The simulator is an algorithm like a prover, but unlike a prover who knows the secret to convince the verifier, the simulator knows no secret. The simulator must fool a Verifier into believing their statement is true while generating a transcript indistinguishable from a Prover.

Because a Simulator has no knowledge but can convince a verifier with an indistinguishable transcript, then every transcript must be indistinguishable and cannot tell the difference between the Prover and the Simulator.

\paragraph{Proving knowledge of a discrete logarithm}
Also known as the Schnorr signature, 

\pcb{
\textbf{ Alice} \< \< \textbf{ Bob}  \\[0.1\baselineskip][\hline]
 \<\< \\[-0.5\baselineskip]
 sk \in \ZZ_q, \pk := g^{\sk}\<\< \\
 s \gets \ZZ_q, T := g^s\<\< \\
 \< \sendmessageright*{\pk, T} \< \\
 \<\< e \sample \ZZ \\
 \< \sendmessageleft*{e} \< \\
z = s + e\cdot\sk \<\< \\
 \< \sendmessageright*{z} \< \\
 \<\< g^{z} \stackrel{?}{=} (\pk)^e \cdot T\\
}

Completeness: 
\begin{equation}
    \begin{aligned}
        g^{z} \stackrel{?}{=}& (\pk)^e \cdot T\\
        g^{s + e\cdot\sk} {=}& (g^{\sk})^e \cdot g^s\\
        g^{s + e\cdot\sk} {=}& g^{s + e\cdot\sk}\\
    \end{aligned}
\end{equation}

\Note{prove soundness and zero knowledgeness for each}
Soundness: to prove soundness, we show a knowledge extractor exists for "every possible prover". We show that by rewinding the prover's protocol, it allows the knowledge extractor to extract the secret key.

\paragraph{Proving knowledge of the opening of a pedersen commitment}

\pcb{
\textbf{ Alice} \< \< \textbf{ Bob}  \\[0.1\baselineskip][\hline]
 \<\< \\[-0.5\baselineskip]
 m \in \ZZ_q, r \gets \ZZ_q \< \< \\
 \commitment := g^mh^r \<\< \\
 s_1, s_2 \gets \ZZ_q^2 \<\< \\
 T := g^{s_1}h^{s_2}\<\< \\
 \< \sendmessageright*{\GG,q,g,h,\commitment, T} \< \\
 \<\< e \sample \ZZ \\
 \< \sendmessageleft*{e} \< \\
z_1 = s_1 + em \<\< \\
z_2 = s_2 + er \<\< \\
 \< \sendmessageright*{z_1, z_2} \< \\
 \<\< g^{z_1}h^{z_2} \stackrel{?}{=} (\commitment)^e \cdot T\\
}
